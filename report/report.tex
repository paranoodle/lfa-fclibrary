\documentclass{article}
\usepackage[utf8]{inputenc}
\usepackage[margin=1in]{geometry}
\usepackage{graphicx}
\bibliographystyle{plainnat}
\usepackage{url}

\setlength{\parskip}{1em}

\title{Fuzzy Clustering Library}
\date{13.5.2016}
\author{Eléonore d'Agostino et Benoît Zuckschwerdt}

\begin{document}
  \pagenumbering{arabic}
  \maketitle
  \tableofcontents
  \newpage
  
  \section{Introduction}
  
  \section{Contexte}
  
    Le \textit{Clustering} consiste à prendre un ensemble de données, et à le diviser en groupes, ou \textit{clusters}.
  
    \subsection{Fuzzy Clustering}
    
      Aussi connu sous le nom de \textit{Soft Clustering}, le Fuzzy Clustering, à la différence du \textit{Hard Clustering}, permet à un point d'appartenir à zero où plusieurs clusters, et d'avoir un degré d'appartenance à chacun des clusters du système.
    
    \subsection{Algorithme K-means}
    
      Algorithme de hard clustering, K-means est l'algorithme de clustering le plus basique. Il génère des fonctions d'appartenance binaires, ne permettant que des valeurs de 1 (appartient au cluster) ou 0 (n'appartient pas au cluster).
      
      \# algo détaillé ici
      
      \subsubsection{K-means++}
    
    \subsection{Algorithme Fuzzy C-means}
    
    \subsection{Algorithmes Expectation-Maximization}
  
  \section{Etat de l'art}
  
  \section{Réalisation}
  
  \section{Conclusion}

\bibliography{biblio}
\nocite{*}

\end{document}
