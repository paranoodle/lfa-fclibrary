\documentclass{article}
\usepackage[utf8]{inputenc}
\usepackage[margin=1in]{geometry}
\usepackage{graphicx}
\bibliographystyle{plainnat}
\usepackage{url}

\setlength{\parskip}{1em}

\title{Fuzzy Clustering Library}
\date{13.5.2016}
\author{Eléonore d'Agostino et Benoît Zuckschwerdt}

\begin{document}
  \pagenumbering{arabic}
  \maketitle
  \tableofcontents
  \newpage

  \section{Introduction}

  \section{Contexte}

    Le \textit{Clustering} consiste à prendre un ensemble de données, et à le diviser en groupes, ou \textit{clusters}.

    \subsection{Fuzzy Clustering}

      Aussi connu sous le nom de \textit{Soft Clustering}, le Fuzzy Clustering, à la différence du \textit{Hard Clustering}, permet à un point d'appartenir à zero où plusieurs clusters, et d'avoir un degré d'appartenance à chacun des clusters du système.

    \subsection{Algorithme K-means}

      Algorithme de hard clustering, K-means est l'algorithme de clustering le plus basique. Il génère des fonctions d'appartenance binaires, ne permettant que des valeurs de 1 (appartient au cluster) ou 0 (n'appartient pas au cluster).

      \# algo détaillé ici

      \subsubsection{K-means++}

    \subsection{Algorithme Fuzzy C-means}

    \subsection{Algorithmes Expectation-Maximization}

  \section{Etat de l'art}

  \subsection{SciPy}

  SciPy est une librairie Python possédant une implémentation de K-means.

  \subsection{scikit-fuzzy}

  scikit-fuzzy est une autre librairie Python, mais cette fois possédant une implémentation de C-means

  \subsection{Mathematica}

  Mathematica est un logiciel de calcul édité par Wolfram Research. Il possède une implémentation de C-means.

  \subsection{MATLAB}

  MATLAB est un language de programmation, il est utilisé à des fins de calculs numériques.

  \

  Ce langage possède une implémentation de C-means,
  mais également une implémentation de Subtractive Clustering
  qui est un algorithme pour éstimer le nombres de clusters pour un jeu de données.

  %Fuzzy C-Means Clustering et Subtractive Clustering -> http://ch.mathworks.com/help/fuzzy/fuzzy-clustering.html?requestedDomain=www.mathworks.com

  \subsection{R}

  R est un logiciel libre de traitement de données et d'analyse statistique, il utilise le langage de programmation S.



  %FANNY -> https://stat.ethz.ch/R-manual/R-devel/library/cluster/html/fanny.html
  % ET: http://www.r-bloggers.com/fuzzy-clustering-with-fanny/

  \section{Réalisation}

  Nous allons développer nos fonctions en Python, ce choix nous semble le plus logique, vu que on utilise ce language dans ce cours.

  Pour commencer nous allons mettre en place K-means dont l'implémentation est plus simple que celle de C-means,
  cela nous permettra également de bien comprendre le fonctionnement des ces méthodes et par la suite de les comparer (C-means étant de la logique floue).

  Nous nous intéresseront ensuite aux algorithmes Expectation-Maximization.

  \section{Conclusion}

\bibliography{biblio}
\nocite{*}

\end{document}
